\documentclass[10pt, compress]{beamer}


\usetheme[usetitleprogressbar]{m}


\usepackage{booktabs}
\usepackage[scale=2]{ccicons}
\usepackage{minted}
\usepgfplotslibrary{dateplot}

\usemintedstyle{trac}
\usepackage{verbatim}

% % %for using default theme comment above lines and decomment following lines
%\usetheme{default}
%\usepackage[utf8]{inputenc}
%\usepackage[T1]{fontenc}

%\AtBeginSection[]
%{
%	\begin{frame}<beamer>
%		\frametitle{Plan}
%		\tableofcontents[currentsection]
%	\end{frame}
%}


\begin{document}
	\title{Neuroscience and network effects}
	\subtitle{Project plan}
	\date{ \vspace{1cm} \hspace{0.1cm}  Giugno 2015}

	\author{\begin{tabular}{l@{ }l} 
			Studenti:      & Emanuele Pesce\\
			& Alessandro Merola \\[1ex]
		\end{tabular}}
		\institute{\small \begin{tabular}{l@{}l}
			\emph{Corso di Reti Neurali e Knowledge discovery} & \\
				  \footnotesize Università degli studi di Salerno, Dipartimento di Informatica & \\
				 & \\
%				\footnotesize Giugno 2015 & \\
			\end{tabular}
			
%			\hspace{0.1cm} Corso di Automi, Linguaggi e Complessità \newline
%					\hspace{0.5cm} Università degli studi di Salerno, Dipartimento di Informatica\\
					}
		
		\begin{frame}
			\titlepage
		\end{frame}
		
%		\begin{frame}{Outline}
%			\tableofcontents
%		\end{frame}
%		
		%**************************************************************************
		
		\section{Definizione obiettivo}
			\begin{frame}{Definizione obiettivo}
				\begin{itemize}
					\item \textbf{Input}: i grafi delle connettività delle regioni di interesse.
					\item \textbf{Obiettivo}: ricavare informazioni strutturali delle reti al fine di discriminare controlli da pazienti.
				\end{itemize}
			\end{frame}
		
		
		\begin{frame}{Thresholding}
			\begin{block}{\textbf{Preprocessing}}
				\begin{itemize}
					\item Attualmente i grafi sono completamente connessi
					\item Come tagliare gli archi meno rilevanti?
				\end{itemize}
			\end{block}
		\end{frame}
		
		%**************************************************************************
		\section{Proposte}
			\begin{frame}{Proposte}
				\begin{block}{\textbf{1. Misure di centralità}}
					Individuare quali sono le regioni più \emph{centrali} in base a diversi criteri. \newline
					\begin{itemize}
						\item Degree
						\item Betweenness
						\item PageRank
						\item Closeness
						\item etc \dots
					\end{itemize}
				\end{block}
			\end{frame}
			
			\begin{frame}{Proposte}
				\begin{block}{\textbf{2. Analisi strutturale del grafo}}
					\begin{itemize}
						\item Distribuzione dei gradi (power low, etc..)
						\item Smallworld
						\item \textbf{community detection}
					\end{itemize}
				\end{block}
			\end{frame}
			
			\begin{frame}{Proposte}
				\begin{block}{\textbf{3. Diffusione delle informazioni}}
					\begin{itemize}
						\item Simulazione di modelli di diffusione delle informazioni per studiare informazioni sul flusso di informazione
						\item Individuazione delle regioni  che permettono una maggiore diffusione di informazioni
					\end{itemize}
				\end{block}
			\end{frame}
			
\plain{Fine}
			
\end{document}